\documentclass[11pt]{article}
\usepackage{geometry}
\linespread{1.02}
\usepackage{url,epsfig}
\usepackage{relsize}
\usepackage{amsmath}
\usepackage{amssymb}
\usepackage{graphicx}
\graphicspath{{./fig/}}

\makeatother\renewcommand{\bibitem}{\vskip 2pt\par\hangindent\parindent\hskip-\parindent}

\DeclareMathOperator{\E}{\mathbb{E}}

% Title, authors, date
\title{\bf Regularizing Bayesian linear models with an informative prior on $R^2$\vspace{.1in}}
\author{Ben Goodrich\footnote{Columbia University} \and Jonah Gabry$^{\ast}$
\and Andrew Gelman$^{\ast}$ \and Matthew Stephens\footnote{University of Chicago}\vspace{.1in}}
\date{25 February 2016\vspace{-.2in}}

% Begin document
\begin{document}
\maketitle
\thispagestyle{empty}

\begin{abstract}
\noindent In this paper we derive an approach for expressing prior beliefs about
the location of the $R^2$, the familiar proportion of variance in the outcome
variable that is attributable to the predictors under a linear model. In
particular, when there are many predictors relative to the number of
observations we would expect the joint prior derived here to work better than
placing independent, heavy-tailed priors on the coefficients, which is  standard
practice in applied Bayesian data analysis but  neither reflects the beliefs of
the researcher nor conveys enough information to stabilize all the computations.
\end{abstract}

\section{Introduction}

Fully making Bayesian estimation of linear models routine for applied
researchers requires prior distributions that work well for any data generated
according to the assumptions of the likelihood function. Most Bayesian
approaches require the researcher to specify a joint prior distribution for the
regression coefficients (and the intercept and error variance), but most applied
researchers have little inclination to specify all these prior distributions
thoughtfully and take a short-cut by specifying one prior distribution that is
taken to apply to all the regression coefficients as if they were independent of
each other (and the intercept and error variance).

In this paper we derive and demonstrate an approach for expressing  prior
beliefs about the location of the $R^2$, the familiar proportion of variance in
the outcome variable that is attributable to the predictors under a linear
model. Since the $R^2$ is a well-understood bounded scalar, it is easy to
specify prior information about it. In particular, when there are many
predictors relative to the number of observations we would expect the joint
prior derived here to work better than placing independent, heavy-tailed priors
on the coefficients, which neither reflects the beliefs of the researcher nor
conveys enough information to stabilize all the computations.


\section{Likelihood}

The likelihood for one observation $y_i$ under a linear model can be written as
the conditionally normal density

$$
f(y_i \,|\, \mu_i, \sigma_\epsilon)
= \frac{1}{\sigma_{\epsilon} \sqrt{2 \pi}}
\exp{\left\{-\frac{1}{2} \left(\frac{y_i - \mu_i}{\sigma_{\epsilon}}\right)^2\right\}},
$$
%
where $\mu_i = \alpha + \mathbf{x}_i^\top \boldsymbol{\beta}$ is a linear
predictor and $\sigma_{\epsilon}$ is the standard deviation of the error in
predicting the outcome. The likelihood of the entire sample is the product of
$N$ individual likelihood contributions and it is well-known that the likelihood
is maximized when the sum-of-squared residuals is minimized. This occurs when

\begin{align*}
\widehat{\boldsymbol{\beta}}
  &= \left(\mathbf{X}^\top \mathbf{X}\right)^{-1} \mathbf{X}^\top \mathbf{y},\\
\widehat{\alpha}
  &= \overline{y} - \overline{\mathbf{x}}^\top \widehat{\boldsymbol{\beta}},\\
\widehat{\sigma}_{\epsilon}^2
  &= \frac{\left(\mathbf{y} - \widehat{\alpha} -
        \mathbf{X} \widehat{\boldsymbol{\beta}}\right)^\top
        \left(\mathbf{y} - \widehat{\alpha} -
        \mathbf{X} \widehat{\boldsymbol{\beta}}\right)}{N},
\end{align*}
%
where $\overline{\mathbf{x}}$ is a vector that contains the sample means of the
$K$ predictors, $\mathbf{X}$ is a $N \times K$ matrix of \emph{centered}
predictors, $\mathbf{y}$ is a $N$-vector of outcomes, and $\overline{y}$ is the
sample mean of the outcome.

Taking a QR decomposition of the design matrix,
$\mathbf{X} = \mathbf{Q}\mathbf{R}$,
where $\mathbf{Q}^\top \mathbf{Q} = \mathbf{I}$ and $\mathbf{R}$ is upper
triangular, we can write the OLS solution for the coefficients as

$$
\widehat{\boldsymbol{\beta}}
= \left(\mathbf{X}^\top \mathbf{X}\right)^{-1} \mathbf{X}^\top \mathbf{y}
= \mathbf{R}^{-1} \mathbf{Q}^\top \mathbf{y}.
$$
%
The QR decomposition is often used for numeric stability reasons (see the
familiar {\tt lm} function in R), but, as we outline below, it is also useful
for thinking about priors in a Bayesian version of the linear model.

\section{Priors}

The key innovation in this paper is the prior for the parameters in the
QR-reparameterized model. To understand this prior, think about the equations
that characterize the maximum likelihood solutions before observing the data on
$\mathbf{X}$ and especially $\mathbf{y}$.

What would the prior distribution of
$\boldsymbol{\theta} = \mathbf{Q}^\top \mathbf{y}$ be?
We can write its $k$-th element as
$$\theta_k = \rho_k \sigma_Y \sqrt{N - 1},$$
where $\rho_k$ is the correlation between the $k$th column of $\mathbf{Q}$ and
the outcome, $\sigma_Y$ is the standard deviation of the outcome, and
$1/\sqrt{N-1}$ is the standard deviation of the $k$ column of $\mathbf{Q}$.
Then let $\boldsymbol{\rho} = \sqrt{R^2}\mathbf{u}$ where $\mathbf{u}$ is a unit
vector that is uniformally distributed on the surface of a hypersphere.
Consequently, $R^2 = \boldsymbol{\rho}^\top \boldsymbol{\rho}$ is the familiar
coefficient of determination for the linear model.

An uninformative prior on $R^2$ would be standard uniform, which is a special
case of a $\mathrm{Beta}(a, b)$ distribution with both shape parameters $a$ and
$b$ equal to $1$. A non-uniform prior on $R^2$ is somewhat analogous to ridge
regression, which is popular in data mining and produces better out-of-sample
predictions than least squares because it penalizes
$\boldsymbol{\beta}^\top \boldsymbol{\beta}$,
usually after standardizing the predictors. In our case, an
informative prior on $R^2$ effectively penalizes
$\boldsymbol{\rho}\top \boldsymbol{\rho}$,
which encourages $\boldsymbol{\beta} = \mathbf{R}^{-1} \boldsymbol{\theta}$
to be closer to the origin.

Lewandowski, Kurowicka, and Joe (2009) derives a distribution for a correlation
matrix that depends on a single shape parameter $\eta > 0$, which implies the
variance of one variable given the remaining $K$ variables is
$\mathrm{Beta}\left(\eta,\frac{K}{2}\right)$. Thus, the $R^2$ is distributed
$\mathrm{Beta}\left(\frac{K}{2},\eta\right)$ and any prior information about the
location of $R^2$, which we will denote $\ell_{R^2}$, can be used to choose a
value of the hyperparameter $\eta$.

Four ways of choosing $\eta$ via the specification of $\ell_{R^2}$ are:

\begin{enumerate}
\item $\ell_{R^2}$ is the prior mode on the $\left(0,1\right)$ interval.

This is only valid if $K \geq 2$ since the mode of a
$\mathrm{Beta}\left(\frac{K}{2},\eta\right)$ distribution is
$\left(\frac{K}{2} - 1\right) / \left(\frac{K}{2} + \eta - 2\right)$
and does not exist if $K < 2$.

\item $\ell_{R^2}$ is the prior mean on the $\left(0,1\right)$ interval, where
the mean of a $\mathrm{Beta}\left(\frac{K}{2}, \eta\right)$ distribution is
$\left(\frac{K}{2}\right) / \left(\frac{K}{2} + \eta\right)$.


\item $\ell_{R^2}$ is the prior median on the $\left(0,1\right)$ interval.

The median of a $\mathrm{Beta}\left(\frac{K}{2},\eta\right)$ distribution is not
available in closed form, but if $K > 2$ it is approximately equal to
$\left(\frac{K}{2} - \frac{1}{3}\right) / \left(\frac{K}{2} +
\eta - \frac{2}{3}\right)$.
Regardless of whether $K > 2$, we can numerically solve for the value of $\eta$
that is consistent with a given prior median utilizing.

\item $\ell_{R^2}$ is some (negative) prior value for
$\E\left(\log R^2\right) = \psi\left(\frac{K}{2}\right) -
\psi\left(\frac{K}{2}+\eta\right)$,
where $\psi\left(\cdot\right)$ is the Digamma function. Again, given a prior
value for the left-hand side it is easy to numerically solve for the
corresponding value of $\eta$.
\end{enumerate}


For $\sigma_y$ we set $\sigma_y = \omega s_y$ where $s_y$ is the sample standard
deviation of the outcome and $\omega > 0$ is an unknown scale parameter to be
estimated. The only prior for $\omega$ that does not contravene Bayes' theorem
in this situation is Jeffreys prior,
$$f\left(\omega\right) \propto \frac{1}{\omega},$$
which is proportional to a Jeffreys prior on the unknown $\sigma_y$,
$$f\left(\sigma_y\right) \propto \frac{1}{\sigma_y}
= \frac{1}{\omega \widehat{\sigma}_y} \propto \frac{1}{\omega}.$$
This parameterization and prior makes it easy to work with any continuous
outcome variable, no matter what its units of measurement are.

We do not need a prior for $\sigma_{\epsilon}$ because our prior beliefs about
$\sigma_{\epsilon} = \omega s_y \sqrt{1 - R^2}$ are already implied by our
priors beliefs about $\omega$ and $R^2$. That only leaves a prior for
$\alpha =
\overline{y} - \overline{\mathbf{x}}^\top \mathbf{R}^{-1} \boldsymbol{\theta}$.
As a default, an improper uniform prior is possible (the posterior is still
proper), but something like a Gaussian prior with mean zero and standard
deviation $\sigma_y / \sqrt{N}$ can also work well.



\section{Posterior}

The previous sections imply a posterior distribution for $\omega$, $\alpha$,
$\mathbf{u}$, and $R^2$. After fitting the model, the parameters of interest can
then be recovered as:
%
\begin{align*}
\sigma_y
  &= \omega s_y \\
\sigma_{\epsilon}
  &= \sigma_y \sqrt{1 - R^2} \\
\boldsymbol{\beta}
  &= \mathbf{R}^{-1} \mathbf{u} \sigma_y \sqrt{R^2 \left(N-1\right)}
\end{align*}

When implementing this model, we actually utilize an improper uniform prior on
$\log \omega$. Consequently, if $\log \omega = 0$, then the marginal standard
deviation of the outcome \emph{implied by the model} is the same as the sample
standard deviation of the outcome. If $\log \omega > 0$, then the marginal
standard deviation of the outcome implied by the model exceeds the sample
standard deviation, so the model overfits the data. If $\log \omega < 0$, then
the marginal standard deviation of the outcome implied by the model is less than
the sample standard deviation, so the model \emph{underfits} the data or that
the data-generating process is nonlinear. Given the regularizing nature of the
prior on $R^2$, a minor underfit would be considered ideal if the goal is to
obtain good out-of-sample predictions. If the model badly underfits or overfits
the data, then the model should be reconsidered.



\section{Conclusion}

Priors can be easy or hard for applied researchers to \emph{specify} and easy or
hard for applied researchers to \emph{conceptualize}. Traditional shortcut
priors on regression coefficients are often used because they are both easy to
specify and to conceptualize. The informative prior on $R^2$ proposed in this
paper is more difficult to conceptualize but with the recent release of the
{\tt rstanarm} R package is equally easy to specify.


\section*{References}

\noindent

\bibitem Gabry, J., and Goodrich, B. (2016). rstanarm: Bayesian Applied
Regression Modeling via Stan. R package version 2.9.0-3.
\url{http://mc-stan.org/interfaces/rstanarm}

\bibitem Guan, Y., and Stephens, M. (2011). Bayesian variable selection
regression for genome-wide association studies, and other large-scale problems.
\emph{Annals of Applied Statistics}. {\bf 5}(3): 1780--1815.

\bibitem Lewandowski, D., Kurowicka D., and Joe, H. (2009). Generating random
correlation matrices based on vines and extended onion method.
\emph{Journal of Multivariate Analysis}. {\bf 100}(9): 1989--2001.



% End document
\end{document}
